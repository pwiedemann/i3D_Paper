\section{Introduction}
%Suppose $S$ is a surface embedded in the Euclidean space $R^3$. Furthermore, here we only consider the simplest topology: the surface $S$ is a topological disk. A topological disk is a surface homeomorphic to the planar unit disk, namely, an orientable, genus zero surface with a single boundary. The surface has an induced Euclidean metric $\mathbf{g}$. Suppose $f$ is a function from the surface to $R$. The harmonic energy of the function is defined as $$ E(f) = \int_S |\nabla_{\mathbf{g}} f |^2 dA $$. The minimizer of the harmonic energy is called a harmonic function. A harmonic function satisfies the Laplace equation \[ \Delta_{\mathbf{g}} f = 0. \] with Dirichlet boundary condition \[ f|_{\partial S} = h, \] where $\partial S$ represents the boundary of the surface, $h$ is a given function. Suppose we choose an isothermal coordiante system $(u,v)$, then the Riemannian metric is represented as \[ \mathbf{g} = e^{2\lambda(u,v)} (du^2+dv^2), \] then the gradient operator is \[ \nabla_{\mathbf{g}} = e^{-\lambda(u,v)} (\frac{\partial}{\partial u},\frac{\partial}{\partial v})^T, \] the area element is \[ dA = e^{2\lambda(u,v)} du \wedge dv, \] and the Laplace-Beltrami operator is \[ \Delta_{\mathbf{g}} = e^{-2\lambda(u,v)} (\frac{\partial^2}{\partial u^2} + \frac{\partial^2}{\partial v^2}). \] Suppose $\phi:S \to D$ is a mapping, such that $\phi(u,v)=(x(u,v),y(u,v))$, both $x(u,v)$ and $y(u,v)$ are harmonic functions, then $\phi$ is called a harmonic map. Intuitively, we can imaginge the surface $S$ is made of rubber sheet, then we flattned the surface onto the planar unit disk, fixing the boundary. The mapping will naturally minimizes the membrane energy (stretching energy), and minizer is the harmonic map. Harmonic map is very useful for engineering applications, one fundamental reason is because harmonic map gives diffeomrophism under some appropriate conditions. Namely the mapping is bijective and smooth. This is formulated as Rado's theorem,\\
%\\
%Rado's Theorem Suppose $\phi: S\to D$ is a harmonic map, $D$ is a planar convex domain. If the restriction of the map on the boundary is a homeomorphism, then the interior mapping is a diffeomorphism.\\
%\\
%%%%%%%%%%
Rendering photo-realistic appearances entails solving the \textit{rendering equation} for each point on an object's surface. This computation can be extremely demanding, especially considering global illumination effects where the problem becomes highly recursive.
\textit{Precomputed Radiance Transfer} (PRT) is a technique intended to overcome this by simplifying the rendering equation but still enabling high-quality renderings for complex illuminations. The quintessence is to perform a single pre-computation step of the light-transport and only evaluate scene relative arithemetic at runtime to achieve real-time.
Classic PRT algorithms function well for static scenes; however, these are destined to fail in dynamic and/or interactive environments, in which considered objects undergo significant deformations.
Most significant is the component of the rendering equation called the \textit{transfer function} which is fully dependent of the shape structure of the object's surface. That is, any object deformation implies a re-computation of the transfer function, often using many ray-traced samples for each surface point. Hence, using classic PRT to render deformable objects would involve pre-computing large amounts of data, leading to immense storage consumption, rapidly becoming unwieldy for such applications.

Additionally, the time and memory consuming pre-computation of these transfer functions, presumes knowledge of all future deformations of the regarded object. Nevertheless, dynamic or interactive scenes may require on-the-fly adaptive, previously unknown, object deformations. 
%For instance, more recent developments in the field of automatic character animations involving on-the-fly pose adaptation \cite{DeepHuman,Holden2017, QuadrupedMotion}. 
%Examples of such are: 
%interactive physically based deformations for cloth or soft-bodies [find references] ; 
%or more recent developements in the field of automatic character animations involving on-the-fly pose adaptation \cite{DeepHuman,Holden2017, QuadrupedMotion}. 

We propose a Deep Learning framework to overcome the limitations of traditional PRT algorithms described above to enable practical use for diversely deforming objects. In particular, we replace expensive ray-tracing algorithms by a deep Convolutional Neural Network (CNN) that, for a given deformation, infers the corresponding set of SH - coefficients that represent the transfer function. 
Thus, we enable a compact PRT representation that maintains a constant/fixed storage consumption, regardless of the number of deformations. Moreover, due to the inherent generalization capabilities of deep Neural Networks, our method is able to accurately predict appearances of previously unknown shapes. We call our approach \textit{Deep Precomputed Radiance Transfer} (DeepPRT). 


Finding an appropriate representation of shape, or manifold like, data to use in a CNN framework is a challenging task due to the non-Euclidean nature of the domain in which the data is defined on. In such domains, basic operations, such as the convolution, are not well defined. Nonetheless, more recently some authors have started to address non-Euclidean data proposing a variety of approaches \cite{ShapeNet1, Geometric_deep_learning, CNN_on_Torus} (for further reading: \cite{GeoDeepLearning}).
In particular, we propose learning on a regular surface representation \textit{geometry images} proposed by Gu et al. \shortcite{gu2002geometry}. Surfaces are mapped into squares, we perform harmonic mapping, and resampled into a regular grid. This regular surface space representation allows standard convolution operations and shows to be advantageous within the DL context due to its dieffeomorphic property, as shown by Sinha et al. \shortcite{sinha2016deep}. 
\\
The main contribution of our approach is: 
\begin{itemize}
\item a method of appearance prediction for PRT enabling more general (\textit{free form}) and adaptive deformations, whilst maintaining a fixed compressed representation.
\end{itemize}

