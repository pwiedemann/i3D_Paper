%\begin{abstract}
%
%%-------------------------------------------------------------------------
%%  ACM CCS 1998
%%  (see http://www.acm.org/about/class/1998)
%% \begin{classification} % according to http:http://www.acm.org/about/class/1998
%% \CCScat{Computer Graphics}{I.3.3}{Picture/Image Generation}{Line and curve generation}
%% \end{classification}
%%-------------------------------------------------------------------------
%%  ACM CCS 2012
%   (see http://www.acm.org/about/class/class/2012)
%%The tool at \url{http://dl.acm.org/ccs.cfm} can be used to generate
%% CCS codes.
%%Example:
%\begin{CCSXML}
%<ccs2012>
%<concept>
%<concept_id>10010147.10010371.10010352.10010381</concept_id>
%<concept_desc>Computing methodologies~Collision detection</concept_desc>
%<concept_significance>300</concept_significance>
%</concept>
%<concept>
%<concept_id>10010583.10010588.10010559</concept_id>
%<concept_desc>Hardware~Sensors and actuators</concept_desc>
%<concept_significance>300</concept_significance>
%</concept>
%<concept>
%<concept_id>10010583.10010584.10010587</concept_id>
%<concept_desc>Hardware~PCB design and layout</concept_desc>
%<concept_significance>100</concept_significance>
%</concept>
%</ccs2012>
%\end{CCSXML}
%
%\ccsdesc[300]{Computing methodologies~Collision detection}
%\ccsdesc[300]{Hardware~Sensors and actuators}
%\ccsdesc[100]{Hardware~PCB design and layout}
%
%
%\printccsdesc   
%\end{abstract}
% --------
% abstract
\begin{abstract}
TODO:\\
(Written by Yue Li)\\
\textit{   Traditional Precomputed Radiance Transfer methods  require a lot of memory to save the precomputed data for real time rendering. For an animation sequence such data would be gigantic. We proposed a deep learning precomputed radiance transfer framework (DPRT) for deforming object, saving only the weights of the network, which lowers the memory cost in orders of magnitudes. Object is first parameterized via harmonic mapping and reconstructed to form geometry and normal image as the inputs of a carefully designed fully convolutional network. }
\end{abstract}

%CCS
\begin{CCSXML}
<ccs2012>
<concept>
<concept_id>10010147.10010371.10010372</concept_id>
<concept_desc>Computing methodologies~Rendering</concept_desc>
<concept_significance>500</concept_significance>
</concept>
<concept>
<concept_id>10010147.10010371.10010372.10010374</concept_id>
<concept_desc>Computing methodologies~Ray tracing</concept_desc>
<concept_significance>500</concept_significance>
</concept>
</ccs2012>
\end{CCSXML}

\ccsdesc[500]{Computing methodologies~Rendering}
\ccsdesc[500]{Computing methodologies~Ray tracing}

%keywords
\keywords{ray tracing, global illumination, octrees, quadtrees}
